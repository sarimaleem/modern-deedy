\documentclass[]{resume-openfont}

\pagestyle{fancy}
\resetHeaderAndFooter

%--------------------------------------------------------------
% Convenience command - make it easy to fill template

% Create job position command. Parameters: company, position, location, when
\newcommand{\resumeHeading}[4]{\runsubsection{\uppercase{#1}}\descript{ | #2}\hfill\location{#3 | #4}\fakeNewLine}

% Create education heading. Parameters: Name, degree, location, when
\newcommand{\educationHeading}[4]{\runsubsection{#1}\hspace*{\fill}  \location{#3 | #4}\\
\descript{#2}\fakeNewLine}

% Create project heading. Parameters: Name, link, Tech stack
\newcommand{\projectHeading}[3]{\Project{#1}{#2}
\descript{#3}\\}

\newcommand{\projectHeadingWithDate}[4]{\Project{#1}{#2}
\descript{#3 | #4}\\}

% Parameters: courses
\newcommand{\courseWork}[1]{\textbf{Coursework:} #1}

% Parameters: courses
\newcommand{\teacherAssistant}[1]{\textbf{Teacher Assistant (TA):} #1}
 
%--------------------------------------------------------------
\begin{document}

%--------------------------------------------------------------
%     Profile
%--------------------------------------------------------------
\newcommand{\yourName}{Sarim Aleem}
% How you want it to show up on the resume
\newcommand{\yourWebsite}{abdullaharif.tech}
% vs how you want it to show up. If it's you can just replace "\yourWebsiteLink" with "yourWebsite"
\newcommand{\yourWebsiteLink}{https://abdullaharif.tech}
\newcommand{\yourEmail}{sarimaleem99@gmail.com}
\newcommand{\yourPhone}{281-804-5071}
\newcommand{\githubUserName}{sarimaleem}
\newcommand{\linkedInUserName}{sarim-aleem}

% \alignProfileTable
% \begin{tabular*}{\textwidth}{l@{\extracolsep{\fill}}r}
%     \ralewayBold{\href{\yourWebsiteLink}{\Large \yourName}} & 
%     Email : \href{mailto:\yourEmail}{\yourEmail}
%     \\
%     \href{https://github.com/\githubUserName}{GitHub://\githubUserName} & 
%     Mobile : \yourPhone
%     \\
%     \href{https://www.linkedin.com/in/\linkedInUserName}{LinkedIn://\linkedInUserName} & Website : \href{\yourWebsiteLink}{\yourWebsite}
%     \\
% \end{tabular*}

\begin{center}
    \Huge \scshape \latoRegular{\yourName} \\ \vspace{1pt}
    \small \href{mailto:\yourEmail}{\underline{\yourEmail}}  $|$  % \yourPhone $|$ 
    \href{https://www.linkedin.com/in/\linkedInUserName}{\underline{linkedIn/\linkedInUserName}} $|$
    \href{https://github.com/\githubUserName}{\underline{github/\githubUserName}} 
    % $|$ \href{\yourWebsiteLink}{\underline{\yourWebsite}}
\end{center}

%--------------------------------------------------------------
%     Education
%--------------------------------------------------------------
\section{Education}
% Put school first and degree second if your school is reputable
\educationHeading{University of Texas at Austin}{B.S. Computer Science, Minor Arabic, GPA: 3.83}{Austin, TX}{May 2024}

% \teacherAssistant{World Wide Web Information Systems Development}
\courseWork{Data Structures, Algorithms, Computer Architecture, Operating Systems, Distributed Systems, Compilers, Computer Networks, Intro to Machine Learning}
\sectionsep

%--------------------------------------------------------------
%     Experience
%--------------------------------------------------------------
\section{Work Experience}
\resumeHeading{Fujitsu Network Communications}{Software Development Intern}{Dallas TX}{Jun 2022 – Aug 2022}
\begin{bullets}
    \item Developed custom webclient to store network topology data in Configuration Persistent Storage database using Java Spring
    \item (alternative?) Migrated network element data storage platform from MongoDB to CPS using Java Spring
    \item Configured Docker files so microservices can communicate with each other in docker-compose
    \item (alternative?) Configured network bridge in Docker files in order for microservices to communicate with each other
    \item Migrated topological discovery of network elements from direct drivers to a software defined network controller
    \item Created bash scripts to automate container deployment and test REST endpoints
    \item Used Java reactive streams to load balance client requests from frontend and ensure requests are nonblocking (sounds fancier than it is)
\end{bullets}
\sectionsep

\resumeHeading{Baylor College of Medicine}{Research Assistant (RA)}{Houston, TX}{Jan 2021 - August 2021}
\begin{bullets}
    % \item Developed a Personal Health Record (PHR) system in \textbf{\href{https://spring.io/}{Spring}} based on the theoretical model outlined in \underlinedLink{https://research.library.mun.ca/11920}{Mitu Kumar's thesis}.
    % \item Applied the \href{https://link.springer.com/chapter/10.1007/978-3-642-10838-9\_23}{mCP-ABE} encryption scheme using the \href{http://gas.dia.unisa.it/projects/jpbc/}{JPBC} library, so patients have fine-grained access control over their health records with the ability to instantly revoke access.
    \item Created computer vision model in C++ to fit ellipses to pupils of mice, improving speed over deep learning model by \textbf{50x}
    \item (alternative?) Created computer vision model in c++ to analyze mice pupil dialation size, faster than deep learning model by \textbf{50x}
    \item Programmed Tkinter GUI to edit audio files based on signal patterns and spectrogram
    \item Developed Python library to analyze pupil dilation of mice 
    \item Analyzed and Visualized Data with Pandas and Matplotlib 
\end{bullets}
\sectionsep

%--------------------------------------------------------------
%     Projects
%--------------------------------------------------------------
\section{Projects}

\projectHeading{Markdown based Content Management System}{https://github.com/sarimaleem/slog}{Go, TypeScript, Express, MongoDB}
\begin{bullets}
    \item Created web server where users can upload markdown content using Go based Command line app
    \item Deployed Web server using GCP that converts markdown to HTML accessible through REST endpoints
\end{bullets}
\sectionsep

\projectHeading{Distributed Key-Value Store with Load Balancing}{}{Java, Distributed Systems}

\begin{bullets}
    \item Created linearizable distributed key value Store which replicates data with fault tolerance using Paxos.
    \item Load balanced key/value workloads using sharded Paxos based replica groups
\end{bullets}
\sectionsep

\projectHeading{Pintos Operating System}{}{C, Concurrency, Operating Systems}
\begin{bullets}
    \item Implemented Operating System with Thread Priority Scheduling, System Calls, Virtual Memory, and EXT based File System
    \item Debugged race conditions, deadlocks, memory errors, and other bugs using GDB
\end{bullets}
\sectionsep

\projectHeading{Multiplayer Gomoku with AI}{https://multitictactoe.herokuapp.com/}{Javascript, Express, Web Sockets}
\begin{bullets}
    \item Created multiplayer website with communication through websockets
    \item Implemented single player mode with AI using Minimax with Alpha Beta Pruning for 
\end{bullets}
\sectionsep

\projectHeading{Sorting Website}{https://sorting-visualizer-1a3ee.firebaseapp.com/}{React.js, Firebase}
\begin{bullets}
    \item Created website to visualize sorting algorithms using React.js, deployed using firebase
\end{bullets}
\sectionsep


%--------------------------------------------------------------
%     Skills
%--------------------------------------------------------------
\section{Skills}
\begin{skillList}
    \singleItem{Languages:}{Java, C, C++, Python, Bash, SQL}
    \\
    \singleItem{Web Development:}{React, JavaScript, TypeScript, HTML/CSS, Express, Node}
    \\
    \singleItem{Technology:}{Git, SQL, Docker, \LaTeX, MongoDB, Spring}
\end{skillList}
\section{Activities}
\singleItem{Central Texas Model United Nations}{Organized Model UN conference for highschoolers across Texas}
\\
\singleItem{Advancing UT Technology}{Team leader for social media website to connect UT students}
\\
\singleItem{Texas Judo}{Participated in Judo contests}
\\
\end{document}
